\documentclass[10pt,letterpaper,titlepage]{article}
\usepackage[latin1]{inputenc}
\usepackage{amsmath}
\usepackage{amsfonts}
\usepackage{amssymb}
\usepackage{gensymb}
\usepackage{geometry}
\usepackage{graphicx}
\usepackage{hyperref}
\usepackage{placeins}
\geometry{letterpaper, portrait, margin=1.25in}
\hypersetup{colorlinks=true,linkcolor=blue,urlcolor=blue}
\graphicspath{{images/}}
\author{Matt Gill}
\begin{document}
	\begin{titlepage}
		\begin{center}
			\vspace*{1cm}
            \includegraphics[width=0.8\textwidth]{accupatt_logo.png}\\
			\vspace{0.5cm}
			\textbf{Aircraft Spray Pattern Testing Software}\\
			User Manual - Version 2.0.x
			\vfill
			\textbf{Developed By:}\\
			Matt Gill - University of Illinois\\
            \vspace{0.5cm}
			\textbf{Based on Previous Work By:}\\
			Dr. Richard Whitney - WRK of Oklahoma\\
			Phil Jank - USDA Aerial Application Technology Research Unit\\
            \vspace{0.5cm}
			\textbf{Aided Significantly by the Technical Expertise Of:}\\
			Dr. Scott Bretthauer - National Agricultural Avaition Association\\
            Dr. Brad Fritz - USDA Aerial Application Technology Research Unit\\
            Dr. Dennis Gardisser - WRK of Arkansas\\
		\end{center}
	\end{titlepage}
	\tableofcontents
	\newpage
	
    \section{Introduction}
	\textit{AccuPatt} is a desktop application created to process spray pattern data for agricultural aircraft. This includes measuring tracer dye deposition on string collectors and measuring droplet deposition from spray cards. The goals of this application are twofold:
    \begin{itemize}
        \item Maximize flexibility through parameterization to provide broad utility
        \item Wherever possible, simplify; The best tools don't need user manuals (And yet here we are)
    \end{itemize}

    \newpage

    \section{Compaitble Hardware}

    \subsection{String Hardware}
    To collect relative fluorescence deposition data from a linear (string) collector, special hardware is required. Specifically: 
    \begin{itemize}
        \item A light source of the excitation wavelength for the intended tracer dye
        \item A means to measure intensity of the emitted wavelength for the intended tracer dye
        \item A controlled means of moving the string through the excitation/emission point
    \end{itemize}
    Currently there are two systems available to accomplish this:
    \begin{itemize}
        \item{\textit{WRK System:}} The \textit{WRK of Oklahoma} String Spectrometer System is commercially available for purchase, and is the system that this software was originally built for. It uses a combination wavelength-specific LED and Band-Pass filter as the excitation source and a commercial USB Spectrometer (Ocean Optics, USB2000+ or Flame) as the emission wavelength measurement device. It utilizes a set of serially controlled stepper motors in conjunction with various clutches and pulleys to move the string at a constant speed. Any questions on the \textit{WRK of Oklahoma} system should be directed to its creator, Dr. Richard Whitney (\href{mailto:whitney3451@att.net}{whitney3451@att.net})
        \item{\textit{USDA-ARS AATRU System:}} The \textit{USDA} String Spectrometer System is not commercially avaialable, however, build specifications, a parts list and build instructions are freely available. It uses a combination wavelength-specific LED and Band-Pass filter as the excitation source and a commercial USB Spectrometer (Ocean Optics, USB2000+ or Flame) as the emission wavelength measurement device. It utilizes a variable-speed controller to drive a lightly-modified portable drill, with a rotary encoder providing feedback for PID control. Any questions on the \textit{USDA} system should be directed to its creator, Phil Jank (\href{mailto:phil.jank@usda.gov}{phil.jank@usda.gov}).
    \end{itemize}
    \textbf{\color{red}Note: Currently, only the WRK System is supported within AccuPatt. Compatibility with the USDA system is planned, but the timeline is uncertain.\color{black}}

    \subsection{Spray-Card Hardware}
    To process spray-cards, they must first be scanned with a flatbed scanner. It is recommended to use a dedicated scanner (not a combination printer/scanner) for the highest quality. A supported resolution of at least 1200 dpi is recommended for most applications. \textit{AccuPatt} does not communicate with the scanner. Instead, images are scanned using the software of your choice (typically that which comes with the scanner itself) and those images can be uploaded to \textit{AccuPatt} for processing.
    
    \newpage

    \section{Installation}

    \subsection{Windows}
    After downloading the installer, double click it to launch the install wizard. You will have the option to create a desktop shortcut if desired.\\
    \\\textbf{\color{red}Note: Windows 10 or newer required.\color{black}}\\\\
    The \textit{WRK System} requires the following drivers to be installed:
    \begin{itemize}
        \item Install the \textit{OmniDriver} Spectrometer hardware driver provided by the manufacturer. When prompted in the install process, choose the Redistributable (free) version. \url{https://www.oceaninsight.com/support/software-downloads/omnidriver-and-spam/}
        \item Install the \textit{FTDI (VCP)} Stepper Motor hardware driver provided by the manufarcturer. \url{https://ftdichip.com/drivers/vcp-drivers/}
    \end{itemize}
    \subsection{MacOS}
    After downloading the .dmg file, double click it to open it as a volume. Drag the \textit{AccuPatt} icon to the \textbf{Applications} shortcut icon. Depending on how you acquired the file, you may be required to right click on the app and choose \textit{Open} to bypass MacOS Gatekeeper settings.\\ 
    \\\textbf{\color{red}Note: MacOS 11 or newer required.\color{black}}\\\\
    The \textit{WRK System} requires no additional drivers.
    \newpage

    \section{Operational Overview}
    
    \subsection{New Series}
    All data and operations in \textit{AccuPatt} require a series. To acquire new data, a new series must be created. From the menubar, use \textbf{File $\rightarrow$ New Series} to view options available (Figure~\ref{fig:new_series}). A series is created with the user-defined default number of passes.
    \begin{itemize}
        \item \textbf{New Series $\rightarrow$ New Aircraft:} Create an empty series with required parameters set to default values.
        \item \textbf{New Series $\rightarrow$ File Aircraft:} Create a series, copying all application info from a previous data file. Increments the series number by 1. A list of previous data files in your current working directory will appear in the menu for you to select from. Alternatively, choosing \textbf{Select File Aircraft} will open a File Chooser allowing you to select a datafile from anywhere.
    \end{itemize}
    \begin{figure}[hb]
        \centering
        \includegraphics[width=0.6\textwidth]{new_series}
        \caption{New Series}
        \label{fig:new_series}
    \end{figure}
    
    \subsection{Application Info}
    It is generally desirable, but not required, to enter in some amount of information about the aircraft, spray system configuration, and/or applicator information to accompany pattern testing data. The \textbf{Application Info} tab provides fields to facilitate this (Figure~\ref{fig:application_info}). See the \hyperref[sec:data]{Data Structure} section for more details on specific fields.
    \\\\
    The text entered in \textbf{Fly-In Info} fields is stored persistently.
    \\\\
    There is button to the right of the \textbf{Pilot} field which, if hovered over, reads \textbf{Load Business Info from File}. Clicking this will prompt to select a previous datafile, from which all fields within \textbf{Applicator Info} section (excluding Pilot) will be filled.
    \\\\
    When an aircraft make and model are selected, the wingspan field is automatically populated if available.
    \\\\
    For fields which contain a unit dropdown selector, the chosen units are stored persistently.
    \\\\
    The \textbf{Nozzle Info} section allows adding any number of unique nozzle sets. 

    \begin{figure}[hb]
        \centering
        \makebox[\textwidth]{\includegraphics[width=\textwidth]{application_info.png}}
        \caption{Application Info Tab}
        \label{fig:application_info}
    \end{figure}
    \FloatBarrier

    \subsubsection{Pass Observables}
    In addition to \textbf{Application Info}, which is collected on a per-series basis, there is some observable data which is typically collected on a per-pass basis. These are frequently referred to here and in the program as \textbf{Pass Observables} since they consist solely of measured values obtained at the time of the application.\par
    These values can be entered in several locations for the convenience of the analyst. They are entered by use of the \textbf{Pass Observables} button on the \textbf{Application Info} tab or alternately from either the \textbf{Capture/Edit String} winodw or the \textbf{Card Manager} window for any individual pass. Choosing \textbf{File $\rightarrow$ Pass Manager} from the menu will open the \textbf{Pass Manager} window (Figure~\ref{fig:pass_manager}) which contains a table with rows for each pass and columns for each pass observable parameter. In addition to the editable fields for pass observables, the pass name itself may be changed here. There are also checkboxes to facilitate the inclusion/exclusion of string and/or card data from series-wise calculations and plots.\par
    The \textbf{Pass Manager} window also provides buttons to add or remove passes from the series. There is no functional limit to the number of passes per series. A checkbox is provided which, if checked, will update the default number of passes to add to all future series.
    \begin{figure}[hb]
        \centering
        \makebox[\textwidth]{\includegraphics[width=\textwidth]{pass_manager.png}}
        \caption{Pass Manager}
        \label{fig:pass_manager}
    \end{figure}
    \newpage

    \subsection{String Analysis}
    String data is collected on a per-pass basis, but may be analyzed on a per-series basis. The \textbf{String Analysis} tab provides a list of passes in the series and pass/series processing options. This tab also contains sub-tabs for viewing/modifying data as discussed below.
    
    \subsubsection{Capture Pass}
    Choose the pass to be collected from the list of passes, then click the \textbf{Capture Pass} button. This will open the \textbf{Capture/Edit Pass} window (Figure~\ref{fig:capture_edit_pass}).\par
    This window provides indicators for the connection status of the string drive and spectrometer, with access to modify their settings (see the \hyperref[sec:string_hardware_settings]{String Hardware System Settings} section for more details). You can also hover over the \textbf{String Drive} and \textbf{Spectrometer} labels to view a configuration summary.\par
    This window also provides optional data entry fields for observables specific to the applicable pass.\par
    If both the string drive and spectrometer are connected, clicking the \textbf{Start} button will engage the string drive, advancing the string, and simultaneously will change its text to \textbf{Mark}. Clicking the \textbf{Mark} button will begin the collection of spectrometer data at intervals specified by the set \textbf{integration time}. Data will continue to be collected until the set \textbf{string length} is reached. At any time after data collection is begun, clicking the \textbf{Abort} button will stop the string drive and clear all data collected up to that point.\par
    The \textbf{Clear} button may be used to clear the pass collected string data whenever data collecation is not in progress. If string data exists for the current pass, it will ask for confirmation first.
    \begin{figure}[h]
        \centering
        \makebox[\textwidth]{\includegraphics[width=\textwidth]{string_edit_pass.png}}
        \caption{Capture/Edit Pass Window}
        \label{fig:capture_edit_pass}
    \end{figure}
    \newpage

    \subsubsection{View \& Trim Pass}
    Once string data for a pass is collected, the pass will have a check next to it in the pass list on the \textbf{String Analysis} tab. When a pass with string data is selected from the pass list, the \textbf{Individual Passes} sub-tab will show two plots (Figure~\ref{fig:string_individual}): 
    \begin{itemize}
        \item \textbf{Raw Data:} [Upper Plot] Plots data exactly as acquired. \textbf{Trim L} and \textbf{Trim R} (yellow) handles can be moved to exclude data outside their limits. The \textbf{Floor} handle will automatically be set to the lowest non-excluded data point, however, it may be further raised to pull up lower data points to a new level. This is typically done for subjectively removing background noise.
        \item \textbf{Modified Data:} [Lower Plot] Plots trimmed and rebased data (white) and overlays that with a copy which has had a smoothing filter applied (yellow). Rebasing and smoothing are only shown when applicable (see \hyperref[sec:string_options]{String Options})
    \end{itemize}
    \begin{figure}[h]
        \centering
        \makebox[\textwidth]{\includegraphics[width=\textwidth]{string_individual.png}}
        \caption{String Analysis - Individual Passes}
        \label{fig:string_individual}
    \end{figure}
    \newpage

    \subsubsection{Composite}
    The \textbf{Composite} sub-tab contains two plots (Figure~\ref{fig:string_composite}):
    \begin{itemize}
        \item \textbf{Overlay:} [Upper Plot] Each individual (non-excluded) pass is plotted together. 
        \item \textbf{Average:} [Lower Plot] Each individual (non-excluded) psss is used to generate an average pattern. A dashed overlay can be optionally plotted based on the respective option in the String \textbf{Plot Options} window.
    \end{itemize}
    \begin{figure}[hb]
        \centering
        \makebox[\textwidth]{\includegraphics[width=\textwidth]{string_composite.png}}
        \caption{String Analysis - Composite}
        \label{fig:string_composite}
    \end{figure}
    \newpage

    \subsubsection{Simulations}
    The \textbf{Simulations} sub-tab contains two plots (Figure~\ref{fig:string_simulations}). Both plots simulate additively overlapping the measured pattern at intervals of the adjusted swath width. The bold black line represents the resultant summed pattern, and is interpreted here as projected in-field uniformity. The dashed horizontal line represents the mean value, about which standard deviations are computed for coefficient of variation (CV) calculations. In other words, the further a data point on the bold black line is (vertically) from the dashed line, the more it contributes to the CV.
    \begin{itemize}
        \item \textbf{Racetrack:} [Uppper Plot] Each simulated pass is plotted going the same direction (into the screen/paper)
        \item \textbf{Back \& Forth:} [Lower Plot] Every odd simulated pass is mirrored about the center. \color{red} The back \& forth simulation will not accurately represent in-field uniformity if the base pattern is obtained in a crosswind. \color{black}
    \end{itemize}
    The number of \textbf{simulated adjascent passes} (each side) can be set to any non-negative integer. The plots may be generated either with only the mathematically relevant portion or with all simulated passes shown based on the respective setting in the String \textbf{Plot Options} window.
    \\\\
    A table is included which calculates the CV for select swath widths near the set adjusted swath width.
    \begin{figure}[hb]
        \centering
        \makebox[\textwidth]{\includegraphics[width=\textwidth]{string_simulations.png}}
        \caption{String Analysis - Simulations}
        \label{fig:string_simulations}
    \end{figure}
    \newpage

    \subsubsection{Options}
    \label{sec:string_options}
    The \textbf{String Analysis} tab contains \textbf{Pass Options} which affect only the selected pass:
    \begin{itemize}
        \item \textbf{Include In Composite:} Once a pass is recorded, a checkmark will appear adjascent to it in the pass list. This indicates that it is recorded and included in composite (series-wise) calculations and plots. Clicking the checkmark will turn it into a dash, indicating that it is recorded, but excluded ffom composite calculations and plots.
        \item \textbf{Smooth:} Apply smoothing filter to pass
        \item \textbf{Align Center:} Shift pattern left/right using chosen center method.
        \item \textbf{Rebase to X-Trim:} Recalculate each data point location such that the trimmed length is equal to the captured string length. If string speed is slightly faster than originally calibrated, this facilitates post-process correction to ensure accurate swath measurement.
        \item \textbf{Advanced Options:} Opens a popup with the following options:
        \begin{itemize}
            \item \textbf{Smoothing Window:} X-domain based distance over which the sliding smoothing filter is fit and applied
            \item \textbf{Smoothing Order:} Polynomial order for smoothing filter
            \item \textbf{Center Method:} Choose whether to center the pattern with the centroid or Center of Distribution.
        \end{itemize}
    \end{itemize}
    The \textbf{String Analysis} tab also contains \textbf{Series Options} which can apply to all passes and/or the average pattern:
    \begin{itemize}
        \item \textbf{Smooth Average:} Apply smoothing filter to average pattern
        \item \textbf{Align Center Average:} Shift average pattern left/right using chosen center method.
        \item \textbf{Equalize Integrals:} Scale each pass to the same integrated area
        \item \textbf{Advanced Options:} Opens a popup with the following options:
        \begin{itemize}
            \item \textbf{Smoothing Window:} X-domain based distance over which the sliding smoothing filter is fit and applied
            \item \textbf{Smoothing Order:} Polynomial order for smoothing filter
            \item \textbf{Center Method:} Choose whether to center the pattern with the centroid or Center of Distribution.
        \end{itemize}
    \end{itemize}
    The \textbf{String Analysis} tab also contains controls for setting the \textbf{Adjusted Swath Width} which defaults to the target swath width, but is saved independently.\\\\
    The \textbf{String Analysis} tab also contains a \textbf{Plot Options} button which opens a popup with the following options. These options have no mathematical effect on the analysis, and are merely ways to adjust the generated graphics.
    \begin{itemize}
        \item \textbf{Average Plot Dashed Overlay:} Enable/Disable
        \begin{itemize}
            \item \textbf{Inner-Swath-Half-Average:} Computes one-half the average y-value within the \textbf{Adjusted Swath Width} range and draws a horizontal line at that value bounded by vertical lines at the bounds of the \textbf{Adjusted Swath Width}. This is sometimes referred to as the "old-school" way of determining effective swath width, where the intersection points of the horizontal and vertical lines should approximately touch the plotted pattern.
            \item \textbf{Average:} Computes the average y-value of the entire pattern and draws a horizontal line at that value.
        \end{itemize}
        \item \textbf{Simulation X Range:} Set the x-bounds of the overlap simulation plots as one of:
        \begin{itemize}
            \item \textbf{One Swath Width:} Only the mathematically relevant portion of the simulated overlap will be shown, specifically datapoints between -swathwidth/2 and swathwidth/2.
            \item \textbf{Sum of All Swath Widths:} All selected simulated adjascent passes will be shown. Can be useful for illustrating the big picture of in-field overlap
        \end{itemize}
    \end{itemize}

    \newpage

    \subsection{Spray-Card Analysis}
    Card data is collected on a per-pass basis, but may be analyzed on a per-card or per-series basis as well. The \textbf{Card Analysis} tab provides a list of passes in the series. This tab also contains sub-tabs for viewing/modifying data as discussed below.

    \subsubsection{Capture Pass}
    \textit{AccuPatt} does not provide any functionality for digitizing spray-cards; instead, images acquired by the scanning software of your choice can be uploaded, saved to the datafile and used for further processing.\par
    In the \textbf{Card Analysis} tab, click the \textbf{Capture/Edit Pass} button to open the \textbf{Card Manager} window (Figure~\ref{fig:card_manager}).\par 
    \begin{figure}[h]
        \centering
        \makebox[\textwidth]{\includegraphics[width=\textwidth]{card_manager.png}}
        \caption{Spray-Card Analysis - Card Manager}
        \label{fig:card_manager}
    \end{figure}
    The \textbf{Card Manager} window provides a table of card data, with each row representing a card and each column a property of that card. properties may be edited by either double-clicking on them or utilizing provided checkboxes as applicable.\par
    Rows may be selected individually or by click-dragging over multiple rows. There are 4 buttons below this table:
    \begin{itemize}
        \item \textbf{Add:} Adds a new card in the last row, with default settings.
        \item \textbf{Remove:} Removes all selected cards. If any selected cards have image data in the datafile, a confirmation will appear prior to removal.
        \item \textbf{Shift Up:} Shift selected cards up by one row, if possible
        \item \textbf{Shift Down:} Shift selected cards down by one row, if possible.
    \end{itemize}
    Below these buttons are controls for utilizing \textbf{Defined Sets} which facilitate adding a pre-configured list of spray-cards to the list. Use the dropdown to select a set and click \textbf{Add Set To Pass} to append the cards to the table. More details can be found in the \hyperref[sec:defined_sets]{Defined Sets} section.\par
    To the right of the card table, under the \textbf{Observables} tab, is a set of text fields for entering in observables for the current pass. These observables are shared between string and card data on the same pass. See the \hyperref[sec:data]{Data Structure} settings for more details.\par
    At the bottom of the window are controls to \textbf{Upload Images} to the selected cards in the table. A dropdown facilitates choice of the upload method and clicking the \textbf{Upload Image(s) to Selection} button will open a file chooser window to select all applicable files. A more detailed description of image upload methods and their processes can be found in the \hyperref[sec:image_upload]{Spray-Card - Image Upload} section.\par
    Once a card has image data associated with it, selecting its row in the table will make a preview image visible in the \textbf{Image} tab at the right of the window (Figure~\ref{fig:card_manager_image}). The image may be viewed as follows:
    \begin{itemize}
        \item \textbf{Original:} The unmodified original image.
        \item \textbf{Outline:} The original image forms the base layer. All stains have their perimeter outlined.
        \item \textbf{Processed:} A blank image with stains drawn (filled) and colored according to the following:
        \begin{itemize}
            \item \textbf{Undersize:} Not counted toward coverage or droplet spectrum analysis.
            \item \textbf{Edge:} Counted toward coverage, but not droplet spectrum analysis.
            \item \textbf{Valid:} Counted toward coverage and droplet spectrum analysis.
        \end{itemize}
    \end{itemize}
    \begin{figure}[hb]
        \centering
        \makebox[\textwidth]{\includegraphics[width=\textwidth]{card_manager_image.png}}
        \caption{Spray-Card Analysis - Card Manager}
        \label{fig:card_manager_image}
    \end{figure}
    The purpose of these image views is to subjectively verify the image processing efficacy. In addition to editing process options in the table directly, clicking the \textbf{Process Options} button will provide a more interactive adjustment experience. For more details, see the \hyperref[sec:image_process]{Image Process Options} section.\par
    In addition to editing spread factor options in the table directly, clicking the \textbf{Spread Factors} button will provide a more interactive adjustment experience.
    Once satisfied with the properties in the card table and with images uploaded, click \textbf{Ok} to return to the main window.
    \FloatBarrier
    \newpage

    \subsubsection{Individual Passes}
    When a pass is selected from the pass list on the \textbf{Card Analysis} tab, the \textbf{Individual Passes} sub-tab contains a location-domained plot and a table of spray-card statistics (Figure~\ref{fig:card_individual_pass}). The table's first several columns are editable, so that card name, location, location units and inclusion in composite flag may be set here. The plot will automatically update according to modifications made in the table. The y-scale of the plot as well as the inclusion of shading are based on the respective options in the Card \textbf{Plot Options} window.

    \begin{figure}[hb]
        \centering
        \makebox[\textwidth]{\includegraphics[width=\textwidth]{card_individual_pass.png}}
        \caption{Spray-Card Analysis - Individual Pass View}
        \label{fig:card_individual_pass}
    \end{figure}
    \FloatBarrier
    \newpage

    \subsubsection{Composite}
    The \textbf{Composite} tab contains two plots (Figure~\ref{fig:card_composite}):
    \begin{itemize}
        \item \textbf{Overlay:} [Uppper Plot] Each individual (non-excluded) pass is plotted together.
        \item \textbf{Average:} [Lower Plot] Each individual (non-excluded) pass is used to generate an average pattern.  A dashed overlay can be optionally plotted based on the respective option in the Card \textbf{Plot Options} window. The plot can optionally include shading based on the respective options in the Card \textbf{Plot Options} window.
    \end{itemize}
    \begin{figure}[hb]
        \centering
        \makebox[\textwidth]{\includegraphics[width=\textwidth]{card_composite.png}}
        \caption{Spray-Card Analysis - Composite View}
        \label{fig:card_composite}
    \end{figure}
    \FloatBarrier
    \newpage

    \subsubsection{Simulations}
    The \textbf{Simulations} sub-tab contains two plots (Figure~\ref{fig:card_simulations}). Both plots simulate additively overlapping the measured pattern at intervals of the adjusted swath width. The bold black line represents the resultant summed pattern, and is interpreted here as projected in-field uniformity. The dashed horizontal line represents the mean value, about which standard deviations are computed for coefficient of variation (CV) calculations. In other words, the further a data point on the bold black line is (vertically) from the dashed line, the more it contributes to the CV.
    \begin{itemize}
        \item \textbf{Racetrack:} [Uppper Plot] Each simulated pass is plotted going the same direction (into the screen/paper)
        \item \textbf{Back \& Forth:} [Lower Plot] Every odd simulated pass is mirrored about the center. \color{red} The back \& forth simulation will not accurately represent in-field uniformity if the base pattern is obtained in a crosswind. \color{black}
    \end{itemize}
    The number of \textbf{simulated adjascent passes} (each side) can be set to any non-negative integer. The plots may be generated either with only the mathematically relevant portion or with all simulated passes shown based on the respective setting in the Card \textbf{Plot Options} window.
    \\\\
    A table is included which calculates the CV for select swath widths near the set adjusted swath width.

    \begin{figure}[hb]
        \centering
        \makebox[\textwidth]{\includegraphics[width=\textwidth]{card_simulations.png}}
        \caption{Spray-Card Analysis - Simulations}
        \label{fig:card_simulations}
    \end{figure}
    \newpage

    \subsubsection{Droplet Distribution}
    The \textbf{Droplet Distribution} sub-tab contains two histogram plots (Figure~\ref{fig:card_droplet_dist}):
    \begin{itemize}
        \item \textbf{Spray Volume Contribution:} [Upper Plot] After discretizing the droplet spectrum, shows how much relative spray volume is contained in droplets of each discrete bin.
        \item \textbf{Quantity:} [Lower Plot] After discretizing the droplet spectrum, shows how many droplets of each discrete bin are found.
    \end{itemize}
    A table to the right of the histogram plots contains some select labeled droplet spectrum statistics.\par
    \color{red} The droplet spectrum category provided by \textit{AccuPatt} is based on reference nozzle laser diffraction measurements by USDA ARS. Because the measurement method is different the provided droplet spectrum category should not be considered absolute.\color{black}\par
    Dropdown menus at the upper-right facilitate selection of the scope of calculation. The first menu allows seleciton of a series-wise composite (disabling the second menu) or selection of a pass. The second menu allows selection of a pass-wise composite (for the above selected pass) or selection of an individual card. Cards and passes which have been actively excluded will not count toward any composite calculations.
    \begin{figure}[hb]
        \centering
        \makebox[\textwidth]{\includegraphics[width=\textwidth]{card_droplet_dist.png}}
        \caption{Spray-Card Analysis - Droplet Distribution}
        \label{fig:card_droplet_dist}
    \end{figure}
    \FloatBarrier
    \newpage

    \subsubsection{Options}
    The \textbf{Card Analysis} tab contains \textbf{Pass Options} which affect only the selected pass:
    \begin{itemize}
        \item \textbf{Align Center:} Shift pattern left/right using chosen center method.
    \end{itemize}
    The \textbf{Card Analysis} tab also contains \textbf{Series Options} which apply only to the average pattern.
    \begin{itemize}
        \item \textbf{Align Center Average:} Shift pattern left/right using chosen center method.
    \end{itemize}
    The \textbf{Card Analysis} tab also contains controls for setting the \textbf{Adjusted Swath Width} which defaults to the target swath width, but is saved independently.\\\\
    The \textbf{Card Analysis} tab also contains a \textbf{Plot Options} button which opens a popup with the following options. These options have no mathematical effect on the analysis, and are merely ways to adjust the generated graphics.
    \begin{itemize}
        \item \textbf{Y-Axis:} Choice of plotting calculated \textbf{Coverage} or \textbf{Deposition}. If \textbf{Deposition} is chosen, units will be either \textit{gal/a} or \textit{l/ha} depending on the \textbf{Target Rate} units selected in the \textbf{Application Info} tab.
        \item \textbf{Shading:} Enable/Disable shading (filling) the plots using the following options:
        \begin{itemize}
            \item \textbf{Color Method:} Currently only \textbf{DSC} method is supported which computes the Droplet Spectrum Category for all points and shows the pseudo-standard color.
            \item \textbf{Interpolation:} For coloration between measured data points, \textbf{Linear} interpolation may be used or simply the \textbf{Nearest} neighbor values. \color{red}Note: \textbf{Linear} interpolation (i.e. interpolating the $D_{V0.1}$ and $D_{V0.5}$ values between measured datapoints) may result in values (i.e. DSC colors) shown between measured datapoints which do not actually exist in the measured data.\color{black}
        \end{itemize}
        \item \textbf{Average Plot Dashed Overlay:} Enable/Disable
        \begin{itemize}
            \item \textbf{Inner-Swath-Half-Average:} Computes one-half the average y-value within the \textbf{Adjusted Swath Width} range and draws a horizontal line at that value bounded by vertical lines at the bounds of the \textbf{Adjusted Swath Width}. This is sometimes referred to as the "old-school" way of determining effective swath width, where the intersection points of the horizontal and vertical lines should approximately touch the plotted pattern.
            \item \textbf{Average:} Computes the average y-value of the entire pattern and draws a horizontal line at that value.
        \end{itemize}
        \item \textbf{Simulation X Range:} Set the x-bounds of the overlap simulation plots as one of:
        \begin{itemize}
            \item \textbf{One Swath Width:} Only the mathematically relevant portion of the simulated overlap will be shown, specifically datapoints between -swathwidth/2 and swathwidth/2.
            \item \textbf{Sum of All Swath Widths:} All selected simulated adjascent passes will be shown. Can be useful for illustrating the big picture of in-field overlap
        \end{itemize}
    \end{itemize}

    \newpage

    \section{Spray-Card - Details}

    \subsection{Defined Sets}
    \label{sec:defined_sets}
    A convenience method for adding spray-cards to a pass is through the use of \textbf{Defined Sets} (Figure~\ref{fig:card_defined_set}). Navigate to the \textbf{Card Analysis} tab $\rightarrow$ \textbf{Add/Edit Cards} $\rightarrow$ \textbf{Define/Edit Sets}.\par
    A list of locally defined sets at the left can be modified via the \textbf{Add Set}, \textbf{Remove Set} and double-clicking the name to rename it. Selecting the set will populate the tableview at right with the cards in the set. Cards may be added, removed and shifted within the set via the respective buttons below the tableview.\par
    For regularly spaced cards, they may be added to the set in batch format by clicking the \textbf{Add Regularly Spaced Cards to Set} button (Figure~\ref{fig:card_defined_set_batch}). Given an initial location, quantity and either step distance or final location, a list of cards can be added to meet those requirements.\par
    All sets are stored on your local machine and will be available whenever you use the program.
    \begin{figure}[hb]
        \centering
        \makebox[\textwidth]{\includegraphics[width=\textwidth]{card_defined_set.png}}
        \caption{Spray-Card Analysis - Defined Set}
        \label{fig:card_defined_set}
    \end{figure}
    \begin{figure}[hb]
        \centering
        \makebox[\textwidth]{\includegraphics[width=\textwidth]{card_defined_set_batch.png}}
        \caption{Spray-Card Analysis - Defined Set - Add Batch}
        \label{fig:card_defined_set_batch}
    \end{figure}
    \FloatBarrier

    \subsection{Image Upload}
    \label{sec:image_upload}
    Images are uploaded to the datafile using the method selected in the \textbf{Image Upload Method} dropdown. With the cards you intend to upload selected in the Card Manager tableview, click \textbf{Upload Image(s) to Selection}. Note that image files must be of type \textit{*.png : Portable Network Graphics} or \textit{*.tif, *.tiff : Tagged Image File Format}. Ensure that images are acquired with no pre-processing, corrections, enhancements or compression.

    \subsubsection{One File, Multiple Cards}
    When multiple cards are acquired in a single large scan, this method will allow cropping of each card and saving individually to the datafile. After selecting the file, the image will be shown with each card's independently draggable, resizable region-of-interest (ROI) (Figure~\ref{fig:card_upload_multiple}).\par 
    The image may be panned or zoomed-upon. To decrease RAM consuption, large images are shown in lower resolution here, but full-resolution ROIs will be cropped out of the original image using a scaled coordinate system. The following options are available to adjust the method for which these ROI's are drawn:
    \begin{itemize}
        \item \textbf{Pixels Per Inch:} Automatically set by the image metadata, but user-adjustable. This is used in conjunction with the image pixel dimensions to show calculated image dimensions. These are provided only to allow verification by the user of proper pixels per inch setting.
        \item \textbf{Flip Horizontal/Vertical:} Flips the entire image about the center axes. Allows for multiple configurations of physical card placement on the scanner glass. 
        \item \textbf{Orientation:} Used in conjunction with order to decide automatic assigning of names to cards. Either vertical or horizontal.
        \item \textbf{Order:} Used in conjunction with orientation to dictate automatic assigning of names to cards. Either Increaseing or decreasing.
        \item \textbf{Sampling Area:} After cards are located, the percentage of width and height to draw the rectangular ROI.
    \end{itemize}
    In the case that ROIs are not automatically drawn properly, the \textbf{Add} and \textbf{Remove} buttons may be used as follows:
    \begin{itemize} 
        \item \textbf{Add:} Click the \textbf{Add} button. The cursor will change to a crosshair. Press the left mouse button on one corner of the spray-card, drag the cursor to the other corner, then release the left mouse button. A new ROI will be drawn (scaled per the \textbf{Sampling Area} set above) and all ROIs will be relabeled as necessary (per the \textbf{Orientation} and \textbf{Order} set above). Note that if your original selection of spray-cards (selected from the Card Manager) already each have ROIs drawn, no more may be drawn.
        \item \textbf{Remove:} First, click an ROI you would like to remove. The \textbf{Remove} button will become enabled and its text will change to "Remove [spray-card.name]." Clicking the button will remove the ROI and all ROIs will be relabeled as necessary (per the \textbf{Orientation} and \textbf{Order} set above).
    \end{itemize}
    \begin{figure}[hb]
        \centering
        \makebox[\textwidth]{\includegraphics[width=\textwidth]{card_upload_multiple.png}}
        \caption{Spray-Card Analysis - Upload Image with Multiple Cards}
        \label{fig:card_upload_multiple}
    \end{figure}
    \FloatBarrier

    \subsubsection{One File per Card}
    For images which are already cropped appropriately, they can be directly uploaded to the datafile (Figure~\ref{fig:card_upload_singles}). Selected files [right column] can be dragged to alter the order to match the proper order of declared card names. An \textbf{Assert DPI} menu allows setting of image DPI as automatic (set by image metadata) or to one of several common values. It may also optionally be supplied via direct-type entry. This setting will affect the upload of each file.
    \begin{figure}[hb]
        \centering
        \makebox[\textwidth]{\includegraphics[width=\textwidth]{card_upload_singles.png}}
        \caption{Spray-Card Analysis - Upload Single Images}
        \label{fig:card_upload_singles}
    \end{figure}
    \FloatBarrier

    \subsection{Image Process Options}
    \label{sec:image_process}
    The \textbf{Image Process Options} window is opened by selecting a card in the card list which has image data, and then clicking the \textbf{Edit Threshold} button. It contains the following options:
    \begin{itemize}
        \item \textbf{Threshold Type:} Dropdown menu to select the type of threshold to use when binarizing the imge. Options are HSB (Hue-Saturation-Brightness) or Grayscale. This choice will limit the threshold options as follows:
        \begin{itemize}
            \item \textbf{HSB:} Binarizes image by pixel filtering the color image by ranges of Hue (0-179), Saturation (0-255) and Brightness (0-255). Each range is independently applied as either a band-pass (checking the \textbf{Pass} checkbox) or band-reject filter. The resultant 3 binary masks are then combined with bitwise AND logic. (Figure~\ref{fig:card_image_process_hsb})
            \item \textbf{Grayscale:} Converts the image to an 8-bit (0-255) grayscale image, then uses a threshold value to binarize the image, where values below the threshold are treated as background and values above the threshold are treated as part of a stain. This threshold may be set manually, or automatically. The automatic method uses Otsu's Threshold Method, and is sufficient for most bimodal images. The threshold chosen will be used as a set value for the manual method and an upper limit for the automatic method (the threshold will be automatically calculated and shown, not to exceed the set value; for fully automatic thresholding, set the value to 255). (Figure~\ref{fig:card_image_process_grayscale})
        \end{itemize}
        \item \textbf{Watershed Segmentation:} Attempt to seperate adjoining stains. This applies a Euclidian Distance Transform to the image, locates local maxima and then runs a segmentation algorithm to split stains containing multiple local maxima.
        \item \textbf{Stain Approximation:} After thresholding and segmentation, optionally attempt to transform each stain using the chosen method.
        \begin{itemize}
            \item \textbf{Minimum Enclosing Circle:} Draw the smallest circle possible which contains the stain.
            \item \textbf{Fit Ellipse:} Enclose the stain in the smallest possible rotated rectangle; draw the largest ellipse possible within it. Stain contours with less than 5 perimeter points are left unchanged.
            \item \textbf{Convex Hull:} Expand the stain such that there is no convexity defects (concave regions) between any perimeter points.
        \end{itemize}
        \item \textbf{Minimum Stain Size:} integer area in pixels below which stains should be treated as background.
    \end{itemize}
    By default, options are saved only to the selected card. Checking the \textbf{Apply To All} option of \textbf{Pass} or \textbf{Series} prior to saving will apply the options set to all applicable other cards and save that to the datafile. Additionally, checking \textbf{Update defaults with selection} will cause any spray cards created in the future to utilize these settings by default.
    
    \begin{figure}[hb]
        \centering
        \makebox[\textwidth]{\includegraphics[width=\textwidth]{card_image_process_grayscale.png}}
        \caption{Spray-Card Analysis - Image Process Options - Grayscale}
        \label{fig:card_image_process_grayscale}
    \end{figure}
    \begin{figure}[hb]
        \centering
        \makebox[\textwidth]{\includegraphics[width=\textwidth]{card_image_process_hsb.png}}
        \caption{Spray-Card Analysis - Image Process Options - HSB}
        \label{fig:card_image_process_hsb}
    \end{figure}
    \FloatBarrier

    \subsection{Stain to Droplet Calculation}
    \label{sec:spread_factors}
    The \textbf{Spread Factors} window (Figure~\ref{fig:card_spread_factors}) is opened by selecting a card in the card list which has image data, and then clicking the \textbf{Spread Factors} button. Spread Factor Equations are optionally applied to estimate the droplet size (Droplet Diameter, $D_D$) which created a given stain (Stain Diameter, $D_S$). The Stain Diameter is determined from the circle of equal projection area of the stain. Two equation types are provided, with adjustable spread factors (\textbf{a}, \textbf{b} and \textbf{c} coefficients):
    \begin{itemize}
        \item \textbf{Adaptive:} $D_D = D_S / (aD_S^2+bD_S+cD_S)$
        \item \textbf{Direct:} $D_D = aD_S^2+bD_S+cD_S$
    \end{itemize}
    By default, options are saved only to the selected card. Checking the \textbf{Apply To All} option of \textbf{Pass} or \textbf{Series} prior to will apply the options set to all applicable other cards and save that to the datafile. Additionally, checking \textbf{Update defaults with selection} will cause any spray cards created in the future to utilize these settings by default.
    \begin{figure}[hb]
        \centering
        \makebox[\textwidth]{\includegraphics[width=\textwidth]{card_spread_factors.png}}
        \caption{Spray-Card Analysis - Spread Factors}
        \label{fig:card_spread_factors}
    \end{figure}
    \FloatBarrier

    \newpage

    \section{String Hardware System Settings}
    \label{sec:string_hardware_settings}

    \subsection{String Drive}
    From the \textbf{Capture/Edit Pass} window, string drive settings can be adjusted by clicking the \textbf{Edit String Drive} button, which opens the applicable window (Figure~\ref{fig:string_drive_settings}). This window contains a dropdown for choosing a serial communication port, and accompanying refresh button to re-populate this list (for example, if a device is plugged-in after opening this window). With a port chosen in the list, that port's manufacturer and product information will be displayed.\par
    There are two fields which allow setting of the \textbf{String Length} and \textbf{String Forward Speed}. The units of string length [ft or m] can also be set here, and this will trigger the units of string forward speed [ft/sec or m/sec] to follow. While the string forward speed can be set manually here, this can also be calculated interactively by clicking the \textbf{Calculate String Speed} button (see below) 
    A \textbf{Direct Command} interface is included to facilitate querying and setting string drive stepper motor paramaters to onboard nonvolatile memory. The Take-Up Reel corresponds to address \textbf{A} and the Supply Reel corresponds to address \textbf{B}. With a command typed into the textfield, clicking the \textbf{Send} button will append a carriage return, encode the message and send it to the stepper motor driver. A string of return text is normally recieved, and is displayed when available. This return text may simply be an echo of your command, or it may contain paramater information. See the stepper motor driver documentation by clicking the \textbf{?} button for specific commands.\par
    For convenience, this window may also be reached from the main menu by choosing \textbf{Extras $\rightarrow$ Shortcut - String Drive}.
    \begin{figure}[hb]
        \centering
        \makebox[\textwidth]{\includegraphics[width=0.90\textwidth]{string_drive_settings.png}}
        \caption{String Drive Settings}
        \label{fig:string_drive_settings}
    \end{figure}
    \newpage
    
    \subsubsection{Calculate String Speed}
    This function will use a known length of string to calculate the forward string speed.
    \begin{enumerate}
        \item Enter the known length of string to test with. This defaults to currently set \textbf{String Length} from the \textbf{Edit String Drive} window, but any value may be used.
        \item Align the string so that the start point (mark) is just to the right of (before) the mirror block.
        \item Click the \textbf{Start} button. This will begin advancing string and start a timer.
        \item Once the end point (mark) passes the mirror block, click the \textbf{Stop} button to stop the string drive.
        \item A calculated speed will be shown using \textit{[known-length / elapsed-time]}.
        \item Click \textbf{OK} to save this speed (or \textbf{Cancel} to not do so) and return to the \textbf{Edit String Drive} window.
    \end{enumerate}
    
    \begin{figure}[hb]
        \centering
        \makebox[\textwidth]{\includegraphics[width=1\textwidth]{string_drive_speed.png}}
        \caption{Interactively Calculate String Drive Speed}
        \label{fig:string_drive_speed}
    \end{figure}
    \FloatBarrier
    \newpage

    \subsection{Spectrometer}
    From the \textbf{Capture/Edit Pass} window, spectrometer settings can be adjusted by clicking the \textbf{Edit Spectrometer} button, which opens the applicable window (Figure~\ref{fig:spectrometer_settings}). This window displays the currently connected spectrometer. An accompanying refresh button is included to re-scan for a spectrometer (for example, if one is plugged-in after opening this window).\\\\
    The \textbf{Dye} can be selected from a dropdown menu, and an accompnaying edit button will open the \textbf{Dye Manager}.\\\\
    The \textbf{Display Units} can be chosen here and dictate the y-axis view in the \textbf{Test Spectrometer} and \textbf{Capture/Edit Pass} window only. 
    \begin{itemize}
        \item \textbf{Absolute} will output 16-bit values directly from the spectrometer (0-65,535)
        \item \textbf{Relative (\%):} will output scaled values from 0-100\% by dividing absolute values by 655.35
    \end{itemize}
    There is also a \textbf{Test Spectrometer} button which will open a popup to do just that (defined further below)
    \\\\
    For convenience, this window may also be reached from the main menu by choosing \textbf{Extras $\rightarrow$ Shortcut - Spectrometer}.

    \begin{figure}[hb]
        \centering
        \makebox[\textwidth]{\includegraphics[width=0.80\textwidth]{spectrometer_settings.png}}
        \caption{Spectrometer Settings}
        \label{fig:spectrometer_settings}
    \end{figure}
    \FloatBarrier
    \newpage

    \subsubsection{Dye Manager}
    An editable list of user-configurable Dyes (Figure~\ref{fig:spectrometer_dye}) corresponding to set values for the following:
    \begin{itemize}
        \item \textbf{Excitation Wavelength (nm):} Target wavelength of excitation source. Currently only used in \textbf{Test Spectrometer}.
        \item \textbf{Emission Wavelength (nm):} Target wavelength for dye emission.
        \item \textbf{Integration Time (milliseconds):} Time to collect light for a spectrometer reading. Think of this as the exposure on a camera; longer times correspond with more detail incorporated in each data frame, but a lower quantity of data frames per unit length of string.
        \item \textbf{Boxcar Width (nm):} Sets the wavelength width to average for each spectrometer reading. A value of 0 means to take a reading only from the spectrometer pixel nearest the target emission wavlength. A value of 5, for example, would have each spectrometer reading be an average of every pixel within 5 nm (+/-2.5 nm) of the target emission wavelength. 
    \end{itemize}

    \begin{figure}[hb]
        \centering
        \makebox[\textwidth]{\includegraphics[width=0.80\textwidth]{spectrometer_dye.png}}
        \caption{Dye Manager}
        \label{fig:spectrometer_dye}
    \end{figure}
    \FloatBarrier
    \newpage

    \subsubsection{Test Spectrometer}
    With a spectrometer connected, this window will show a live readout of the entire measured spectrum. The dye selected in the \textbf{Edit Spectrometer} window is used to set target excitation/emission wavelengths as well as integration time and boxcar width. In addition to the plot, live values are shown for relevant measurements. The scale of the y-axis uses the \textbf{Display Units} option set in the \textbf{Spectrometer Settings} window.

    \begin{figure}[hb]
        \centering
        \makebox[\textwidth]{\includegraphics[width=0.95\textwidth]{spectrometer_test.png}}
        \caption{Test Spectrometer}
        \label{fig:spectrometer_test}
    \end{figure}
    \FloatBarrier
    
    \newpage

    \section{Creating PDF Reports}
    Creating a printable/distributable report is a key function, and \textit{AccuPatt} provides utility to generate various report permuations using the \textit{Portable Document Format (*.pdf)}. The following options can be set from the \textbf{Report} menu:
    \begin{itemize}
        \item \textbf{Include Card Images:} [True/False] Whether to add page(s) to the report with individual card images/statistics.
        \item \textbf{Card Image Type:} [Original, Outline, Mask] What type of image to use in report.
        \item \textbf{Card Images Per Page:} [5, 7, 9] Card images are scaled to fit, so this effectively allows balancing desired image size and resultant number of pages required.
        \item \textbf{Include Logo:} If enabled, a chosen image will be added within the report headers.
    \end{itemize}

    \subsection{Operation S.A.F.E. Report}
    Creates a standardized report with the following pages:
    \begin{itemize}
        \item \textbf{String Summary Page:} Headers with all reported application info and observables. Plots overlay, average and simulations. Includes a uniformity (Coefficient of Variation) table. Page will only be included if at least one string pass is collected and selected for inclusion in the composite.
        \item \textbf{Card Summary Page:} Headers with all reported application info and observables. Plots coverage and droplet spectrum category by location, spray volume by droplet diameter and number of droplets by droplet diameter. Pass-wise droplet spectrum characteristics are shown for measured data as well as a USDA Model simulation for the provided nozzle/observable parameters. One page will be included for each pass which has card data associated with it and is selected for inclusion in the composite.
        \item \textbf{Card Image Page:} Individual card images displayed as per the report settings. Individual card statistics shown below each card. Pages included for each pass which has card data associated with it and is selected for inclusion in the composite. Number of pages required will vary based on number of cards and report options selected.
    \end{itemize}
    \newpage

    \section{Exporting Data to Excel}
    While typically unnecessary, it is sometimes desirable to export data obtained in \textit{AccuPatt} into a format more widely accesible. To facilitate this, data can be copied to a \textit{Microsoft Excel (*.xlsx)} file by choosing one of the following options from the \textbf{Export} menu:
    \begin{itemize}
        \item \textbf{Operation S.A.F.E. Attendee Log:} Generates the file required by NAAREF to document attendance at a S.A.F.E. Fly-In. Can choose whether to select desired files or to select a directory (folder) from which all \textit{*.db} files will be included.
        \item \textbf{All Raw Data:} Exactly like it sounds, no processing or calculations, just the data and parameter settings.
        \item \textbf{Processed Data:} \color{red} To-Do \color{black}
        \item \textbf{All Images:} \color{red} To-Do \color{black}
    \end{itemize}
    \newpage

    \section{Opening Past Datafiles}
    From the menubar choose \textbf{File $\rightarrow$ Open} to open a filechooser window and select the file to open. Legacy files created in AccuPatt 1.xx, when opened, will cause a choice prompt to open the file as \textbf{View Only} or to \textbf{Create Compatible Copy} (Figure~\ref{fig:open}). If the latter is chosen, an identically named .db file will be placed in the same directory as the original file. The original (.xlsx) file is never edited, only read.
    \begin{figure}[hb]
        \centering
        \makebox[\textwidth]{\includegraphics[width=0.75\textwidth]{open.png}}
        \caption{Open Datafile}
        \label{fig:open}
    \end{figure}
    \newpage

    \section{Data Structure}
    \label{sec:data}
    To the greatest extent practical, all provided analysis options are persistent for a saved data file. This is to ensure that, when re-opened at a later date or on another computer, results will be as close to identical as possible. Below is a relational outline for a data file.
    \begin{itemize}
        \item \textbf{series-id:} unique identifier, automatically generated.
        \item \textbf{series-number:} integer
        \item \textbf{flyin:}
        \begin{itemize}
            \item \textbf{name:} text
            \item \textbf{location:} text
            \item \textbf{date:} integer, unix timestamp
            \item \textbf{analyst:} text
        \end{itemize}
        \item \textbf{applicator:} 
        \begin{itemize}
            \item \textbf{pilot:} text
            \item \textbf{business:} text
            \item \textbf{street:} text
            \item \textbf{city:} text
            \item \textbf{state:} text
            \item \textbf{zip:} text
            \item \textbf{phone:} text
            \item \textbf{email:} text
        \end{itemize}
        \item \textbf{aircraft:} 
        \begin{itemize}
            \item \textbf{registration:} text
            \item \textbf{make:} text
            \item \textbf{model:} text
            \item \textbf{wingspan:} float, units=[ft, m]
            \item \textbf{winglets:} true/false
        \end{itemize}
        \item \textbf{spray-system:} 
        \begin{itemize}
            \item \textbf{swath-target:} float, units=[ft, m]
            \item \textbf{rate-target:} float, units=[gpa, l/ha]
            \item \textbf{boom-pressure:} float, units=[psi, bar, kpa]
            \item \textbf{boom-width:} float, units=[ft, m]
            \item \textbf{boom-drop:} float, units=[in, cm]
            \item \textbf{nozzle-spacing:} float, units=[in, cm]
        \end{itemize}
        \item \textbf{nozzles:} [list]
        \begin{itemize}
            \item \textbf{type:} text
            \item \textbf{size:} text
            \item \textbf{deflection:} text
            \item \textbf{quantity:} integer
        \end{itemize}
        \item \textbf{notes-setup:} text
        \item \textbf{notes-analyst:} text, will not print on reports
        \item \textbf{string-settings:} 
        \begin{itemize}
            \item \textbf{equalize-integrals:} true/false, scale all passes to largest integrated area
            \item \textbf{average-center:} true/false, apply average-center-method
            \item \textbf{average-center-method:} text, option=[centroid, center-of-distribution]
            \item \textbf{average-smooth:} true/false, apply smoothing filter to average pattern
            \item \textbf{average-smooth-window} float, x-domain window over which to apply smoothing
            \item \textbf{average-smooth-order} integer, polynomial order for smoothing filter
            \item \textbf{number-simulated-adjascent-passes:} integer, per-side basis
        \end{itemize}
        \item \textbf{spray-card-settings:} 
        \begin{itemize}
            \item \textbf{average-center:} true/false, apply average-center-method
            \item \textbf{average-center-method:} text, option=[centroid, center-of-distribution]
        \end{itemize}
        \item \textbf{passes:} [list]
        \begin{itemize}
            \item \textbf{pass-id:} unique identifier, automatically generated
            \item \textbf{name:} editable text, default is 'Pass ' + position in passes list at creation
            \item \textbf{observables:} 
            \begin{itemize}
                \item \textbf{ground-speed:} float, units=[mph, kph, kn]
                \item \textbf{spray-height:} float, units=[ft, m]
                \item \textbf{pass-heading:} integer, range=[0-359]
                \item \textbf{wind-direction:} integer, range=[0-359]
                \item \textbf{wind-speed:} float, units=[mph, kph, kn]
                \item \textbf{temperature:} float, units=[\degree F, \degree C]
                \item \textbf{humidity:} float
            \end{itemize}
            \item \textbf{string:} 
            \begin{itemize}
                \item \textbf{wavelength-excitation:} integer, units=[nm]
                \item \textbf{wavelength-emission:} integer, units=[nm]
                \item \textbf{integration-time:} integer, units=[ms]
                \item \textbf{data-excitation:} list[float, float], [location, intensity]
                \item \textbf{data-emission:} list[float, float], [location, intensity]
                \item \textbf{data-location-units:} units=[ft, m]
                \item \textbf{trim-left:} integer, number of sample points to set to floor value
                \item \textbf{trim-right:} integer, number of sample points to set to floor value
                \item \textbf{trim-vertical:} float, value to subtract from all points in addition to floor value
                \item \textbf{rebase:} true/false, use trim-l/trim-r to scale x-domain 
                \item \textbf{center:} true/false, apply center-method to pass
                \item \textbf{center-method:} text, option=[centroid, center-of-distribution]
                \item \textbf{smooth:} true/false, apply smoothing filter to pass
                \item \textbf{smooth-window} float, x-domain window over which to apply smoothing
                \item \textbf{smooth-order} integer, polynomial order for smoothing filter
                \item \textbf{swath-adjusted:} float, units follow swath-target
                \item \textbf{include-in-composite:} True/False, include pass in series-wise calculations/plots
            \end{itemize}
            \item \textbf{spray-cards:}
            \begin{itemize}
                \item \textbf{center:} true/false, apply center-method to pass
                \item \textbf{center-method:} text, option=[centroid, center-of-distribution]
                \item \textbf{swath-adjusted:} float, units follow swath-target
                \item \textbf{include-in-composite:} True/False, include pass in series-wise calculations/plots
                \item \textbf{spray-cards:} [list]
                \begin{itemize}
                    \item \textbf{id:} unique identifier, automatically generated
                    \item \textbf{name:} text
                    \item \textbf{include-in-composite} true/false, include spray-card in pass-wise calculations/plots
                    \item \textbf{location:} float, optional linear location of spray-card, units=[ft,m]
                    \item \textbf{ppi:} integer, image resolution in pixels per inch, symmetric only
                    \item \textbf{threshold-type:} text, options=[grayscale, hue-saturation-brightness]
                    \item \textbf{threshold-grayscale-method:} text, options=[automatic, manual]
                    \item \textbf{threshold-grayscale:} integer, range=[0-255]
                    \item \textbf{threshold-hsb-hue-range:} [integer, integer], range=[0-179]
                    \item \textbf{threshold-hsb-hue-pass:} true/false, band pass/reject
                    \item \textbf{threshold-hsb-saturation-range:} [integer, integer], range=[0-255]
                    \item \textbf{threshold-hsb-saturation-pass:} true/false, band pass/reject
                    \item \textbf{threshold-hsb-brightness-range:} [integer, integer], range=[0-255]
                    \item \textbf{threshold-hsb-brightness-pass:} true/false, band pass/reject
                    \item \textbf{watershed:} true/false, stain segmentation using euclidian distance map
                    \item \textbf{min-stain-area:} integer, minimum square pixel area to count as stain
                    \item \textbf{stain-approximation-method:} text, options=[none, min-enclosing-circle, fit-ellipse, convex-hull]
                    \item \textbf{spread-factor-equation:} text, options=[none, direct, adaptive]
                    \item \textbf{spread-factor-a:} float
                    \item \textbf{spread-factor-b:} float
                    \item \textbf{spread-factor-c:} float
                    \item \textbf{image:} image, limited to *.png *.tif *.tiff
                \end{itemize}
            \end{itemize}
        \end{itemize}
    \end{itemize}

\end{document}